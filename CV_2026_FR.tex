\documentclass[letterpaper,10pt]{article}

\usepackage{latexsym}
\usepackage[empty]{fullpage}
\usepackage{titlesec}
\usepackage{marvosym}
\usepackage[usenames,dvipsnames]{color}
\usepackage{verbatim}
\usepackage{enumitem}
\usepackage[hidelinks]{hyperref}
\usepackage{fancyhdr}
\usepackage[english]{babel}
\usepackage{tabularx}
\usepackage{multicol}
\input{glyphtounicode}

\usepackage[default]{sourcesanspro}
\usepackage[T1]{fontenc}
%\usepackage{changepage}
\usepackage{graphicx}

\renewcommand{\bfseries}{\fontseries{sb}\selectfont} 

\pagestyle{fancy}
\fancyhf{} 
\fancyfoot{}
\renewcommand{\headrulewidth}{0pt}
\renewcommand{\footrulewidth}{0pt}


\addtolength{\oddsidemargin}{-0.5in}
\addtolength{\evensidemargin}{-0.5in}
\addtolength{\textwidth}{1in}
\addtolength{\topmargin}{-0.75in}
\addtolength{\textheight}{1.25in}

\titleformat{\section}{
  \vspace{2pt}\centering\bfseries
}{}{0em}{}[\color{black}\titlerule\vspace{4pt}]

\setlength{\footskip}{10pt}

\urlstyle{same}

\raggedbottom
\raggedright
\setlength{\tabcolsep}{0in}

\setlength{\multicolsep}{0pt}

\titleformat{\section}{
  \vspace{4pt}\centering\bfseries
}{}{0em}{}[\color{black}\titlerule\vspace{4pt}]

\pdfgentounicode=1

\newcommand{\resumeItem}[1]{
  \item\small{
    {#1 \vspace{-2pt}}
  }
}

\newcommand{\resumeSubheading}[4]{
  \vspace{-2pt}\item
    \begin{tabular*}{0.97\textwidth}[t]{l@{\extracolsep{\fill}}r}
      \textbf{#1} & #2 \\
      \textit{\small#3} & \textit{\small #4} \\
    \end{tabular*}\vspace{-7pt}
}

\newcommand{\resumeSubSubheading}[2]{
    \item
    \begin{tabular*}{0.97\textwidth}{l@{\extracolsep{\fill}}r}
      \textit{\small#1} & \textit{\small #2} \\
    \end{tabular*}\vspace{-7pt}
}

\newcommand{\resumeProjectHeading}[2]{
    \item
    \begin{tabular*}{0.97\textwidth}{l@{\extracolsep{\fill}}r}
      \small#1 & #2 \\
    \end{tabular*}\vspace{-7pt}
}

\newcommand{\resumeSubItem}[1]{\resumeItem{#1}\vspace{-4pt}}

\renewcommand\labelitemii{$\vcenter{\hbox{\tiny$\bullet$}}$}

\newcommand{\resumeSubHeadingListStart}{\begin{itemize}[leftmargin=0.15in, label={}]}
\newcommand{\resumeSubHeadingListEnd}{\end{itemize}}
\newcommand{\resumeItemListStart}{\begin{itemize}}
\newcommand{\resumeItemListEnd}{\end{itemize}\vspace{-5pt}}

\begin{document}

\begin{tabularx}{\textwidth}{@{}m{3cm} X r@{}}
    \includegraphics[width=3cm]{photo2.jpg} &

    \centering
    \begin{tabular}{@{}c@{}}
	    {\LARGE \textbf{Antoine Lopez}} \\[4pt]
	    {\large Data Science/ Artificial Intelligence}
    \end{tabular} &

    \begin{tabular}{@{}r@{}}
	    \href{mailto:{antoine.lopezsimeon@free.fr}}{antoine.lopezsimeon@free.fr}\\
	    +33 6 50 43 88 03 \\
	    Issy-les-Moulineaux, 92130, France \\
	    \href{{https://github.com/C3L0}}{github.com/C3L0}\\
	    \href{{linkedin.com/in/antoinelopez26}}{linkedin.com/in/antoinelopez26}
    \end{tabular}
\end{tabularx}

\vspace{2pt} 
\noindent Futur \textbf{Data/AI Scientist} (Ingénieur en Data/IA) à la recherche d'un stage de fin d'études de 6 mois pour mettre à profit mes compétences en analyse avancée et en projets d'IA dans un environnement stimulant.

\vspace{-10pt}

%-----------EXPERIENCE-----------
\section{Expériences professionnelles}
  \resumeSubHeadingListStart
    \resumeSubheading
      {Stage data engineer }{mai -- aout .2025}
      {Société Générale}{Val de Fontenay, Fr}
      \resumeItemListStart
        \resumeItem{Conception et évaluation des stratégies de test logiciel existantes dans un environement complexe}
        \resumeItem{Proposition et implémentation de nouvelles adaptées aux besoins (Scala, Java, Spark, Apache Kafka)}
        \resumeItem{Développement d’une application de test pour l’ingestion de données dans un datalake cloud, incluant la mise en place d’API, d’un bucket S3 et de pipelines de traitement (Python, API REST, AWS S3)}
    \resumeItemListEnd

    \resumeSubheading
      {Stage développeur logiciel}{janvier -- février .2024}
      {VINCI Energies - Cloud, Communication and Services}{La Défense, Fr}
      \resumeItemListStart
        \resumeItem{Développement d’un Chatbot pour la gestion des tickets clients, utilisant Llama 2 (Python, Docker, VM Linux)}
    \resumeItemListEnd

    \resumeSubheading
      {Stage aide technicien de bureau d’étude}{décembre -- février .2023}
      {VINCI Energies - AXIANS}{Palaiseau, Fr}
      \resumeItemListStart
        \resumeItem{Rédaction de la procédure et traitement des Déclarations de Travaux et des Déclarations d’Intention de Commencement de Travaux}
        \resumeItem{Mise en place d’une base de données sur Excel pour la gestion des contrats clients (Excel, VBA)}
    \resumeItemListEnd

  \resumeSubHeadingListEnd

%-----------Éducation-----------
\section{Éducation}
\resumeSubHeadingListStart

  \resumeSubheading
  {ECE, Paris}{2021 -- Présent}
  {Diplôme d'ingénieur en Big Data / Intelligence Artificielle — Option Cloud}{5\textsuperscript{e} année d’école d’ingénieur (3\textsuperscript{e} année du cycle ingénieur)}

  \resumeSubheading
  {Ajou University, Suwon (Corée du Sud)}{août 2023 -- janvier 2024}
  {Semestre d’échange universitaire — Science informatique}{\,}

  \resumeSubheading
  {Baccalauréat général (Maths, Physique-Chimie, NSI) — Mention Bien}{2021}{}{\,}
  \vspace{-12pt}
\resumeSubHeadingListEnd

%-----------Projets-----------
\section{Projets}
  \resumeSubHeadingListStart

    %\resumeProjectHeading
    %  {\textbf{Analyse \& Visualisation de Base de Données en Python}}{}
    %  \resumeItemListStart
    %    \resumeItem{Utilisation de Matplotlib / Seaborn pour une analyse immobilière de la ville de Lille.}
    %  \resumeItemListEnd

    \resumeProjectHeading
      {\textbf{Projet de fin d'Etude X Alten}}{}
      \resumeItemListStart
        \resumeItem{Développement d'une solution d'optimisation de l'analyse par éléments finis (FEA) avec IA.}
    \resumeItemListEnd

    \resumeProjectHeading
      {\textbf{Projet pluridisciplinaire en Intelligence Artificielle}}{}
      \resumeItemListStart
        \resumeItem{Développement d’un système multi-modèles d’IA pour le résumé, la catégorisation et la traduction d’articles de presse.}
      \resumeItemListEnd

    \resumeProjectHeading
      {\textbf{Hackathon AWS organisé par IPPON}}{}
      \resumeItemListStart
        \resumeItem{Chaotic scenario : Maintien et mise à l'échelle d'une plateforme sur le cloud AWS}
      \resumeItemListEnd
     
%     \resumeProjectHeading
%       {\textbf{Traitement Big Data avec PySpark}}{}
%       \resumeItemListStart
%         \resumeItem{Étude des données du NASDAQ stock à grande échelle.}
%       \resumeItemListEnd

    % \resumeProjectHeading
    %   {\textbf{Functional Programming pour l’Extraction de Données}}{}
    %   \resumeItemListStart
    %     \resumeItem{Développement en Scala pour l’étude du trafic aérien sur une année complète.}
    %   \resumeItemListEnd

    \resumeProjectHeading
      {\textbf{Manipulation de Bases de Données en SQL/NoSQL}}{}
      \resumeItemListStart
        \resumeItem{Utilisation de Oracle SQL / MongoDB / Neo4j pour l’étude des comportements d’achat sur une plateforme en ligne.}
      \resumeItemListEnd

  \resumeSubHeadingListEnd

%-----------Compétences-----------
\section{Compétences}

{\small
    \begin{tabularx}{\textwidth}{@{}X X@{}}
	\textbf{Langages :} Python, Java, C, Scala, SQL& \textbf{Cloud/DevOps :} Azure, AWS, Docker, Git/GitHub\\
	\textbf{IA :} ML, DL, GenAI, NLP, Recommended systems & \textbf{Big Data :} Spark, Kafka, Hadoop, DataBricks, RDBMS, MongoDB, Neo4J\\
	\textbf{Méthode :} Cycle en V, Agile & \textbf{Certification :} AWS Academy Graduate - Cloud Foundations - Training Badge\\
	 \textbf{Visualisation :} Power BI, matplotlib/seaborn& \textbf{Autres compétences :} Réseaux, Robotique, Web 
    \end{tabularx}
    \vspace{-12pt}
}
 
 %-----------Langues-----------
\section{Langues}
\vspace{-6pt}
{\small
\begin{tabularx}{\textwidth}{@{}X X@{}}
      \textbf{Français: } Native & \textbf{Anglais: } Professionnel \\
      \textbf{Espagnol: } Indépendant & \textbf{Coréen: } Débutant \\
    \end{tabularx}
  }
\end{itemize}

%-----------Centre d'intérêts-----------
\section{Centre d'intérêts}
\vspace{-6pt}
{\small
\begin{tabularx}{\textwidth}{@{}X X@{}}	\textbf{Tech: } Suivi des innovations en Intelligence Artificielle et Blockchain & \textbf{Art: } Dessin, Visite de musées et expositions \\
	\textbf{Sports: } Course, Randonnée, Cyclisme, Football, Tennis, Ski, Escalade & \textbf{Jeux vidéo: } Suivi de compétitions e-sport
	 
    \end{tabularx}
  }
\end{itemize}

\end{document}
